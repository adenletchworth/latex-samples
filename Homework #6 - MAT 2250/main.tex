\documentclass{article}
\usepackage[utf8]{inputenc}
\usepackage[paper=letterpaper,margin=2cm]{geometry}
\usepackage{amsmath}
\usepackage{amssymb}
\usepackage{amsfonts}
\usepackage{newtxtext, newtxmath}
\usepackage{enumitem}
\usepackage{titling}
\usepackage[colorlinks=true]{hyperref}

\title{Homework \#6 - MAT 2250}
\author{Aden Letchworth}
\date{Due: December 1, 2022}

\begin{document}

\maketitle

\section*{Chapter 1.1}
    \begin{gather}\tag{\textbf{5}}
    \left[1+\left(\frac{dy}{dx}\right)^2\right] = C 
    \end{gather}
    \begin{center}
        This is an ordinary differential equation or ODE because there aren't partial derivatives with respect to more than one independent variable. The order is first-order because that is the highest-order derivative present. The equation is non-linear because of the quadratic present. The independent variables are \{dx\} and the dependent are \{dy\}.
    \end{center}
\section*{Chapter 1.2} 
    \begin{gather*}\tag{\textbf{4}}
        x = 2cos(t)-3sin(t) ,\hspace{.2cm}x'' + x = 0 \\ \\
            x' = -2sin(t)-3cos(t) \\
            x'' = -2cos(t)+3sin(t)\\
            (-2cos(t)+3sin(t)) + (2cos(t)-3sin(t)) = 0\\
            (2cos(t)-2cos(t)) + (3sin(t)-3sin(t)) = 0\\
            0 + 0 = 0
    \end{gather*}
    \begin{center}
    So, x = 2cos(t)-3sin(t) is a solution to the initial value problem
    \end{center}
    \begin{gather*}\tag{\textbf{10}}
        y-ln(y)=x^2-1, \hspace{.2cm} \frac{dy}{dx} = \frac{2xy}{y-1} \\\\
        \frac{d}{dx}[y] - \frac{d}{dx}[ln(y)] = \frac{d}{dx}[x^2] - \frac{d}{dx}[-1]\\
        \frac{dy}{dx} - \frac{1}{y} \times \frac{dy}{dx} = 2x\\
        \frac{dy}{dx} (1-\frac{1}{y}) = 2x\\
        y(\frac{dy}{dx} (1-\frac{1}{y})) = y(2x)\\
        \frac{dy}{dx}(y-1) = 2xy\\
        \frac{\frac{dy}{dx}(y-1)}{y-1} = \frac{2xy}{y-1}\\
        \frac{dy}{dx} = \frac{2xy}{y-1}\\
    \end{gather*}
    \begin{center}
        So, $y-ln(y)=x^2-1$ is a solution to our equation
    \end{center}
    \begin{gather*}\tag{\textbf{26}}
        \frac{dx}{dt} + cos(x) = sin(t),\hspace{.5cm}x(\pi)=0\\\\
        \frac{dx}{dt} = sin(t)-cos(x)\\
        f(t,x) = sin(t)-cos(x)\\
        \frac{\partial f}{\partial x} = sin(x)
    \end{gather*}
    \begin{center}
        By the existence and uniqueness theorom we know f(t,x) is continuous given the initial values $x(\pi)=0$
    \end{center}
\section*{Chapter 2.2}
    \begin{gather*}\tag{\textbf{8}}
        \frac{dx}{dt}=3xt^2\\\\
        dx[p(x)] = dt[g(t)]\\
        \int\frac{1}{3x}dx=\int t^2dt\\
        \frac{1}{3}\ln(x) = \frac{t^3}{3}+C\\
        \ln|x| = t^3 + 3C\\
        \ln|x| = t^3 + K\\
        x = e^{t^3+K}
    \end{gather*}
    \begin{gather*}\tag{\textbf{22}}
        x^2dx+2ydy=0,\hspace{.5cm} y(0) = 2\\\\
        x^2dx = -2ydy\\
        \int x^2dx = \int =2ydy\\
        \frac{x^3}{3} = -y^2 + C\\
        \frac{(0)^3}{3} = -(2)^2 + C\\
        C = 4\\
        \frac{x^3}{3} = -y^2 + 4\\
        y = \sqrt{\frac{-x^3}{3}+4}\\\\\\\\\\\\\\\\
    \end{gather*}
    \begin{gather*}\tag{\textbf{30}}
    \end{gather*}
    \begin{center}
    As stated in this section, the separation of equation (2) on page 42 requires division by p(y), and this may disguise the fact that the roots of the equation p(y) = 0 are actually constant solutions to the differential equation. 
    \end{center}
    \begin{gather*}\tag{a}
        \frac{dy}{dx} = (x-3)(y+1)^\frac{2}{3}\\\\
        dx[x-3] = dy\left[\frac{1}{(y+1)^\frac{2}{3}}\right]\\
        \int (x-3)dx = \int \frac{1}{(y+1)^\frac{2}{3}}dy\\
        \frac{x^2}{2}-3x=3(y+1)^\frac{1}{3}\\
        \frac{x^2}{6}-x+C = (y+1)^\frac{1}{3} \\
        (y+1) = \left(\frac{x^2}{6}-x+C\right)^3\\
        y = -1+\left(\frac{x^2}{6}-x+C\right)^3\\\\ \tag{b}
        \frac{dy}{dx} = (x-3)(y+1)^\frac{2}{3}\\
        0 = (x-3)(-1+1)\\
        0 = 0 \\
    \end{gather*}
\section*{Chapter 2.3}
    \begin{gather*}\tag{\textbf{17}}
        \frac{dy}{dx}-\frac{y}{x}=xe^x;\hspace{.5cm}y(1) = e-1\\\\
        \mu(x)=e^{\int-\frac{1}{x}dx}\\
        \mu(x)=e^{-ln|x|}\\
        \mu(x) = \frac{1}{x}\\
        \mu(x) \frac{dy}{dx} - \mu(x) \frac{y}{x} = \mu(x)xe^x\\
        \frac{1}{x}\frac{dy}{dx} -\frac{1}{x}\frac{y}{x}=\frac{1}{x}xe^x\\
        \frac{1}{x}\frac{dy}{dx}-\frac{1}{x^2}y=e^x\\
        \int dx\left[\frac{1}{x}y\right]= \int e^x\\
        \frac{y}{x} = e^x + C\\
        y = e^x + Cx\\
        (e-1) = e^{(1)}+C(1)\\
        C = -1\\\\
        y=xe^x - x\\
    \end{gather*}
\pagebreak\section*{Chapter 3.2}
    \begin{gather*}
        \tag{\textbf{8}}
    \end{gather*}
\begin{center}
A tank initially contains $s_0$ lb of salt dissolved in 200 gal of water, where $s_0$ is some positive number. Starting at time t = 0, water containing 0.5 lb of salt per gallon enters the tank at a rate of 4 gal/min, and the well stirred solution leaves the tank at the same rate. Letting c(t) be the concentration of salt in the tank at time t, show that the limiting concentration-that is, $\lim_{t\to\infty}c(t)$-is 0.5lb/gal
\end{center}
\begin{gather*}
    \left(4\frac{gal}{min}\right)\left(.5\frac{lb}{gal}\right) = 2\frac{lb}{min}\\
    4\frac{gal}{min}\left[\frac{x(t)}{200}\frac{lb}{gal}\right]=\frac{x(t)}{50}\frac{lb}{min}\\\\
    \frac{dx}{dt}= 2-\frac{x}{50}\\
    \frac{dx}{dt}=\frac{100-x}{50}\\
    \int \frac{dx}{100-x} = \int \frac{dt}{50}\\
    -ln|100-x| = \frac{1}{50}t+C\\
    100-x = e^{-\frac{1}{50}t-C}\\
    x = -e^{-\frac{1}{50}t-C} + 100
    0 = -e^{-\frac{1}{50}0-C} + 100\\\\
    C = -ln|100-x|-\frac{1}{50}t\\
    C = -ln[100-(0)]-\frac{1}{50}(0)\\
    C = -ln[100]\\\\
    x = -e^{-\frac{1}{50}t-(-ln|100|)} + 100\\
    x = -e^{-\frac{1}{50}t} * e^{ln|100|}+100\\
    x = -100e^{-\frac{1}{50}t}+100\\\\
    \lim_{t\to\infty}x(t) = \lim_{t\to\infty}-100e^{-\frac{1}{50}t}+100\\
    \lim_{t\to\infty}x(t) = 100\\
    \frac{x(t)}{200} = \frac{100}{200}\\
\end{gather*}
\begin{center}
    This shows us that the limiting concentration of salt in the tank is 1/2 or .5 $\frac{lb}{gal}$.
\end{center}
\newpage\section*{Chapter 4.1}
\begin{gather*}\tag{\textbf{7}}
    y''+2y'+4y=5sin3t, \hspace{.5cm} \Omega=3\\\\
    y = Acos\Omega t + Bsin\Omega t\\
    y = Acos3t + Bsin3t\\
    y' = -3Asin3t + 3Bcos3t\\
    y'' = -9Acos3t - 9Bsin3t\\\\
    (-9Acos3t-9Bsin3t)+2(-3Asin3t+3Bcos3t)+4(Acos3t+Bsin3t) = 5sin3t\\
    -9Acos3t-9Bsin3t-6Asin3t+6Bcos3t+4Acos3t+4Bsin3t=5sin3t\\
    -5Acos3t-5Bsin3t-6Asin3t+6Bcos3t = 5sin3t\\
    -5A+6B=0\\
    -5B+6A = 5\\
    A = -\frac{30}{61},\hspace{.5cm}
    B = -\frac{25}{61}\\\\
    y = \left(-\frac{30}{61}\right)cos3t+\left(-\frac{25}{61}\right)sin3t
\end{gather*}
\newpage\section*{Chapter 4.2}
\begin{gather*}\tag{\textbf{20}}
    y''-4y'+4y=0; \hspace{.5cm} y(1) = 1,\;y'(1)=1\\\\
    r^2-4r+4=0\\
    r = 2\\
    y(t) = c_1e^{rt}+c_2te^{rt}\\
    y(t) = c_1e^{2t}+c_2te^{2t}\\  
    y'= 2c_1e^{2t}+2c_2te^{2t}+c_2e^{2t}\\
    1 = c_1e^{2(1)}+c_2(1)e^{2(1)}\\
    1=c_1e^2+c_2e^2\\
    1 = 2c_1e^{2(1)}+2c_2(1)e^{2(1)}+c_2e^{2(1)}\\
    1 = 2c_1e^2+2c_2e^2+c_2e^2\\
    c_1 = \frac{2}{e^2},\;\;\;\;
    c_2 = -\frac{1}{e^2}\\\\
    y(t)=2e^{2t-2}-te^{2t-2}\\\\\\\\\\\\
    \tag{\textbf{29}}
    y_1(t)=te^{2t}, \hspace{.5cm} y_2(t) = e^{2t}\\\\
    y'_1(t) = 2te^{2t} + e^{2t}\\
    y'_2(t) = 2e^{2t}\\\\
    W(y_1,y_2,)(t) = 
    \begin{vmatrix}
        y_1(t) & y_2(t)\\
        y'(t) & y'_2(t)
    \end{vmatrix}\\\\
    W(y_1,y_2)(t) =
    \begin{vmatrix}
        te^{2t} & e^{2t}\\
        2te^{2t}+e^{2t} & 2e^{2t}
    \end{vmatrix}\\\\
    W(y_1,y_2)(t) = ((te^{2t})(2e^{2t}))-((2te^{2t}+e^{2t})(e^{2t})\\
    W(y_1,y_2)(t) = -e^{4t}\\
    -e^{4t} \neq 0\\
\end{gather*}
\begin{center}
The Wronskian of our functions $ y_1 $ and $ y_2 $  cannot equal zero therefore we can say they are linearly independent on the interval (0,1)\\
\end{center}
\newpage
\section*{Chapter 4.3}
\begin{gather*}\tag{\textbf{12}}
    u''+7u = 0\\\\
    r^2+7 = 0\\
    r = i\sqrt{7}, \; -i\sqrt{7}\\
    \alpha = 0, \hspace{.5cm} \beta = \sqrt{7}\\
    u(t)=c_1e^{\alpha t}cos\beta t + c_2e^{\alpha t}sin\beta t\\\\
    u(t) = c_1cos\sqrt{7}t+c_2sin\sqrt{7}t\\\\\\\\\\\\\\\\
    \tag{\textbf{23}}
    w'' - 4w' +2w = 0; \hspace{.5cm}
    w(0) = 2, \hspace{.1cm} w'(0) = 1\\\\
    r^2-4r+2=0\\
    r = 2 + \sqrt{2},\hspace{.2cm} 2 -\sqrt{2}\\
    w(t)=c_1e^{rt}+c_2e^{rt}\\
    w(t) = c_1e^{(2+\sqrt{2})t}+c_2e^{(2-\sqrt{2})t}\\
    w'(t) = (2+\sqrt{2})c_1e^{(2+\sqrt{2})t} + (2-\sqrt{2})c_2e^{(2-\sqrt{2})t}\\
    0 = c_1e^{(2+\sqrt{2})(0)}+c_2e^{(2-\sqrt{2})(0)}\\
    c_1+c_2 = 0\\
    1 = (2+\sqrt{2})c_1e^{(2+\sqrt{2})0} + (2-\sqrt{2})c_2e^{(2-\sqrt{2})0}\\
    1 = (2+\sqrt{2})c_1 + (2-\sqrt{2})c_2\\
    c_1 = \frac{\sqrt{2}}{4}, \hspace{.2cm}c_2 = -\frac{2}{4}\\\\
    w(t) = \frac{\sqrt{2}}{4}e^{(2+\sqrt{2})t}-\frac{\sqrt{2}}{4}e^{(2-\sqrt{2})t}\\
\end{gather*}
\end{document}
